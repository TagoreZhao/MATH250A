\documentclass[12pt]{article}
\usepackage{amsmath, amssymb, amsthm}
\usepackage{geometry}

% Page setup
\geometry{a4paper, margin=1in}

% Title and author
\title{Homework Assignment 2}
\author{Tagore Zhao}
\date{MATH250A \\ Instructor: Richard Borcherds \\ Due Date: \today}

\begin{document}

\maketitle

\section*{Problem 12}
Let \( G \) be a group, and let \( H, N \) be subgroups of \( G \) with \( N \) normal in \( G \). Let \( \gamma_x \) be the conjugation by an element \( x \in G \).

\subsection*{(a)}
Show that \( x \mapsto \gamma_x \) induces a homomorphism \( f: H \to \text{Aut}(N) \).

\subsection*{Solution}

Since \( N \) is normal in \( G \), for any \( n \in N \) and \( g \in G \), we have:
\[
\gamma_g(n) = gng^{-1} \in N.
\]
This implies that \( \gamma_g \in \text{Aut}(N) \) for all \( g \in G \). 

Define \( f: H \to \text{Aut}(N) \) by \( f(h) = \gamma_h \). Since \( \gamma_h \in \text{Aut}(N) \) for every \( h \in H \), the function \( f \) is well-defined.

We need to prove that \( f \) is a homomorphism, i.e., 
\[
f(h_1h_2) = f(h_1)f(h_2).
\]

To show this, we need to demonstrate that \( \gamma_{h_1h_2} = \gamma_{h_1} \circ \gamma_{h_2} \).

For any element \( n \in N \):
\[
\gamma_{h_1h_2}(n) = (h_1h_2) n (h_1h_2)^{-1} = h_1 (h_2 n h_2^{-1}) h_1^{-1} = \gamma_{h_1}(\gamma_{h_2}(n)).
\]
Thus,
\[
\gamma_{h_1h_2} = \gamma_{h_1} \circ \gamma_{h_2}.
\]
Therefore, \( f(h_1h_2) = \gamma_{h_1h_2} = \gamma_{h_1} \circ \gamma_{h_2} = f(h_1) \circ f(h_2) \), showing that \( f \) is indeed a homomorphism.

\subsection*{(b)}
Given subgroups \( H \) and \( N \) of \( G \) with \( N \) normal in \( G \) and \( H \cap N = \{e\} \), show that the map \( H \times N \to HN \) given by \( (x, y) \mapsto xy \) is a bijection, and that this map is an isomorphism if and only if \( f \) is trivial, i.e., \( f(x) = \text{id}_N \) for all \( x \in H \).

\subsection*{Solution}

Define \( H \cap N = \{e\} \) and consider the mapping \( f: H \times N \to HN \) given by \( f(x, y) = xy \).

To show injectivity, suppose \( f(x_1, y_1) = f(x_2, y_2) \). Then:
\[
x_1 y_1 = x_2 y_2.
\]
Multiplying both sides on the right by \( y_2^{-1} \) and on the left by \( x_1^{-1} \), we get:
\[
x_1^{-1} x_2 = y_1 y_2^{-1}.
\]
Since \( x_1^{-1} x_2 \in H \) and \( y_1 y_2^{-1} \in N \), and \( H \cap N = \{e\} \), we must have \( x_1^{-1} x_2 = e \) and \( y_1 y_2^{-1} = e \). Thus, \( x_1 = x_2 \) and \( y_1 = y_2 \), proving that \( f \) is injective.

Next, to show surjectivity, let \( g \in HN \). By definition, \( HN \) consists of all products of the form \( xy \) with \( x \in H \) and \( y \in N \). Thus, every element of \( HN \) can be expressed as the product of an element in \( H \) and an element in \( N \), showing that \( f \) is surjective.

Since \( f \) is both injective and surjective, it is bijective.

Now, we move forward to show that the map is an isomorphism if and only if \( f \) is trivial, i.e., \( f(x) = \text{id}_N \) for all \( x \in H \).

Assume \( f \) is an isomorphism, meaning it is bijective and a homomorphism. For any \( x_1, x_2 \in H \) and \( y_1, y_2 \in N \):
\[
f((x_1, y_1) (x_2, y_2)) = f(x_1 x_2, y_1 y_2) = x_1 x_2 y_1 y_2.
\]
Since \( f \) preserves the group operation, we have:
\[
f((x_1, y_1)) f((x_2, y_2)) = x_1 y_1 x_2 y_2 = x_1 x_2 y_1 y_2.
\]
To prove that \( f \) is trivial, we need to show that \( \gamma_x = \text{id}_N \) for all \( x \in H \), i.e., \( x n x^{-1} = n \) for all \( n \in N \). Since \( H \cap N = \{e\} \), this condition implies that the conjugation action of \( H \) on \( N \) is trivial, confirming the structure is similar to a direct product.

Conversely, if \( f \) is trivial with \( f(x) = \text{id}_N \) for all \( x \in H \), then \( f \) preserves the structure of \( HN \) as a direct-like product, fulfilling the conditions of an isomorphism.

We conclude that \( f \) is an isomorphism if and only if it is trivial.


\subsection*{(c)}
Let \( N, H \) be groups, and let \( \psi: H \to \text{Aut}(N) \) be a given homomorphism. Construct a semidirect product as follows. Let \( G \) be the set of pairs \( (x, h) \) with \( x \in N \) and \( h \in H \). Define the composition law:
\[
(x_1, h_1) (x_2, h_2) = (x_1 \psi(h_1)(x_2), h_1 h_2).
\]
Show that this is a group law, and yields a semidirect product of \( N \) and \( H \), identifying \( N \) with the set of elements \( (x, 1) \) and \( H \) with the set of elements \( (1, h) \).

\subsection*{Solution}

Given groups \( N, H \) and a homomorphism mapping \( \psi: H \to \text{Aut}(N) \), define the composition law:
\[
(x_1, h_1) (x_2, h_2) = (x_1 \psi(h_1)(x_2), h_1 h_2),
\]
for all \( (x_1, h_1), (x_2, h_2) \in G \) with \( x_1, x_2 \in N \) and \( h_1, h_2 \in H \). We aim to show that this is a group law.

\subsubsection*{Closure}
Define \( \psi: H \to \text{Aut}(N) \) by \( \psi(h)(x) = \gamma_h(x) \), where \( \gamma_h \) is the conjugation by \( h \). We first show that \( G \) is closed under this composition:

Let \( (x_1, h_1), (x_2, h_2) \in G \) with \( x_1, x_2 \in N \) and \( h_1, h_2 \in H \). Then,
\[
(x_1, h_1) (x_2, h_2) = (x_1 \psi(h_1)(x_2), h_1 h_2) \in G,
\]
since \( x_1 \in N, \psi(h_1)(x_2) \in N \), and \( h_1 h_2 \in H \). Therefore, \( (x_1 \psi(h_1)(x_2), h_1 h_2) \in G \).

\subsubsection*{Associativity}
To prove associativity, we compute:
\[
((x_1, h_1)(x_2, h_2))(x_3, h_3) = (x_1 \psi(h_1)(x_2), h_1 h_2)(x_3, h_3) = (x_1 \psi(h_1)(x_2) \psi(h_1 h_2)(x_3), h_1 h_2 h_3).
\]
Similarly, 
\[
(x_1, h_1)((x_2, h_2)(x_3, h_3)) = (x_1, h_1)(x_2 \psi(h_2)(x_3), h_2 h_3) = (x_1 \psi(h_1)(x_2 \psi(h_2)(x_3)), h_1 h_2 h_3).
\]
Since \( \psi \) is a homomorphism, \( \psi(h_1 h_2)(x_3) = \psi(h_1)(\psi(h_2)(x_3)) \). Thus, 
\[
((x_1, h_1)(x_2, h_2))(x_3, h_3) = (x_1, h_1)((x_2, h_2)(x_3, h_3)),
\]
showing associativity.

\subsubsection*{Identity}
The identity element in \( G \) is \( (e, e) \), where \( e \) is the identity in both \( N \) and \( H \). We show this by letting \( (x, h) \in G \):
\[
(e, e)(x, h) = (e \psi(e)(x), eh) = (x, h), \quad (x, h)(e, e) = (x \psi(h)(e), he) = (x, h).
\]
Thus, \( (e, e) \) acts as the identity.

\subsubsection*{Inverse}
We want to show that for every \( (x, h) \in G \), there exists \( (x', h') \in G \) such that:
\[
(x, h)(x', h') = (e, e).
\]
Let \( (x, h) \in G \). For the inverse to exist, set:
\[
(x, h)(\psi(h^{-1})(x^{-1}), h^{-1}) = (x \psi(h)(\psi(h^{-1})(x^{-1})), hh^{-1}) = (e, e).
\]
Thus, the inverse of \( (x, h) \) is \( (\psi(h^{-1})(x^{-1}), h^{-1}) \).

\subsection*{Semidirect Product}
We have shown that this composition law indeed defines a group. We now demonstrate that the mapping \( H \times N \to H \ltimes_\psi N \) defined by the above composition law is a semidirect product.

Define \( N \) as the set of elements \( (x, 1) \) and \( H \) as the set of elements \( (1, h) \). It is clear that \( (x, 1), (1, h) \in G \) and:
\[
(x, 1)(1, h) = (x \psi(1)(1), h) = (x, h).
\]
Hence, \( G = NH \).

Next, we show \( N \triangleleft G \). We need to prove that for any \( (x, 1) \in N \) and \( (x', h) \in G \):
\[
(x', h)(x, 1)(x', h)^{-1} = (x', h)(x, 1)(\psi(h^{-1})(x'^{-1}), h^{-1}).
\]
Simplifying:
\[
= (x' \psi(h)(x), h)(\psi(h^{-1})(x'^{-1}), h^{-1}) = (x' \psi(h)(x) \psi(h)(x'^{-1}), e) = (x, e).
\]
Since \( N \) is invariant under conjugation, \( N \triangleleft G \). Therefore, \( G \) is a semidirect product of \( N \) and \( H \).

\section*{Problem 13}
(a) Let \( H, N \) be normal subgroups of a finite group \( G \). Assume that the orders of \( H \) and \( N \) are relatively prime. Prove that \( xy = yx \) for all \( x \in H \) and \( y \in N \), and that \( H \times N \cong HN \).

(b) Let \( H_1, \ldots, H_r \) be normal subgroups of \( G \) such that the order of \( H_i \) is relatively prime to the order of \( H_j \) for \( i \neq j \). Prove that 
\[
H_1 \times \ldots \times H_r \cong H_1 \cdots H_r.
\]

\subsection*{Solution to (a)}
Let \( H, N \triangleleft G \) with \( |H| = p \) and \( |N| = q \), where \(\gcd(p, q) = 1\).

Since \( H \triangleleft G \) and \( N \triangleleft G \), for all \( g \in G \), we have \( gHg^{-1} = H \) and \( gNg^{-1} = N \). Thus, \( H \) and \( N \) are normal subgroups of \( G \), and therefore \( NH = HN \). 

Also, \( H \cap N \triangleleft G \). By Lagrange's Theorem, since \( p \) and \( q \) are coprime, it follows that \( |H \cap N| = 1 \). Hence, \( H \cap N = \{e\} \).

We also know that \( |HN| = |H||N| = pq \), and by Lagrange's Theorem, this implies \( |HN| = |H \times N| \). Therefore, the map \( H \times N \to HN \) given by \( (x, y) \mapsto xy \) is a bijective homomorphism, hence an isomorphism. Thus, we have:
\[
H \times N \cong HN.
\]

Now, we show commutativity. Pick \( x \in N \) and \( y \in H \). Consider \( xyx^{-1}y^{-1} \). Since \( H \triangleleft G \), we have:
\[
xyx^{-1} \in H, \quad y^{-1} \in H, \quad \text{thus } xyx^{-1}y^{-1} \in H.
\]
Similarly, since \( N \triangleleft G \), 
\[
y^{-1}xy \in N, \quad x \in N, \quad \text{thus } xyx^{-1}y^{-1} \in N.
\]
Since \( H \cap N = \{e\} \), it follows that:
\[
xyx^{-1}y^{-1} = e \implies xy = yx.
\]
Thus, \( x \) and \( y \) commute for all \( x \in H \) and \( y \in N \), completing the proof.

\subsection*{Solution to (b)}
Let \( H_1, H_2, \ldots, H_r \) be normal subgroups of \( G \). Assume the order of \( H_i \) is relatively prime to the order of \( H_j \) for \( i \neq j \). Thus, \( H_i \cap H_j = \{e\} \) for \( i \neq j \). The intersection of \( H_1, \ldots, H_r \) contains only the identity element.

Define the map \( \phi: H_1 \times \ldots \times H_r \to H_1 \ldots H_r \) given by:
\[
\phi(h_1, \ldots, h_r) = h_1 \cdots h_r.
\]
To prove injectivity, let \( \phi(h_1, \ldots, h_r) = e \). Then, since \( h_i \in H_i \) and each \( H_i \cap H_j = \{e\} \) for \( i \neq j \), it follows that \( h_1 = h_2 = \ldots = h_r = e \). Therefore, the kernel of \( \phi \) is trivial, implying that \( \phi \) is injective.

For surjectivity, every element in \( H_1 \ldots H_r \) can be expressed as a product \( h_1 \cdots h_r \) with \( h_i \in H_i \). Thus, \( \phi \) is surjective.

Let \( h_i \in H_i \), \( h_j \in H_j \), for any \( i, j \) with \( i \neq j \). Consider the expression \( h_i h_j h_i^{-1} h_j^{-1} \).

Since \( H_j \triangleleft G \), we have:
\[
h_i h_j h_i^{-1} \in H_j, \quad \text{and} \quad h_j^{-1} \in H_j, \quad \text{thus } h_i h_j h_i^{-1} h_j^{-1} \in H_j.
\]

Similarly, since \( H_i \triangleleft G \), we have:
\[
h_j h_i h_j^{-1} \in H_i, \quad \text{and} \quad h_i^{-1} \in H_i, \quad \text{thus } h_i h_j h_i^{-1} h_j^{-1} \in H_i.
\]

Therefore, 
\[
h_i h_j h_i^{-1} h_j^{-1} \in H_i \cap H_j.
\]

Since \( H_i \cap H_j = \{e\} \), it follows that:
\[
h_i h_j h_i^{-1} h_j^{-1} = e,
\]
which implies \( h_i h_j = h_j h_i \).

To prove that \( \phi \) is a homomorphism, consider:
\begin{align*}
    \phi((h_1, \ldots, h_r)(h_1', \ldots, h_r')) &= \phi(h_1 h_1', \ldots, h_r h_r') \\
    &= h_1 h_1' \cdots h_r h_r' \\
    &= h_1 h_2 \cdots h_{r-1}' h_r' \\
    &= \phi(h_1, \ldots, h_r) \phi(h_1', \ldots, h_r').
\end{align*}

Since \( \phi \) is both injective and surjective, it is a bijective homomorphism, hence:
\[
H_1 \times \ldots \times H_r \cong H_1 \cdots H_r.
\]

\section*{Problem 19}
Let \( G \) be a finite group operating on a finite set \( S \).

\subsection*{(a)}
For each \( s \in S \), show that 
\[
\sum_{t \in Gs} \frac{1}{\#(G_t)} = 1.
\]

\subsubsection*{Solution}
We first clarify some notation: let \( \#(G_t) = |G_t| = |G_s| \), where \( G_s \) is the stabilizer of \( s \). Define:
\[
Gs = \{g \cdot s \mid g \in G\}, \quad \text{and} \quad G_t = \{g t \mid t \in Gs, g \in G\}.
\]
For \( t \in Gs \), \( g t = g(g' s) = Gs \) for some \( g' \in G \). We can conclude that \( Gs = G_t \). 

Thus, we have:
\[
\sum_{t \in Gs} \frac{1}{|G_t|} = \frac{1}{|G_s|} \sum_{s \in Gs} 1 = \frac{|Gs|}{|G_s|} = 1.
\]

\subsection*{(b)}
For each \( x \in G \), define \( f(x) \) as the number of elements \( s \in S \) such that \( x s = s \). Prove that the number of orbits of \( G \) in \( S \) is equal to:
\[
\frac{1}{\#(G)} \sum_{x \in G} f(x).
\]

\subsubsection*{Solution}
We denote the number of orbits of \( G \) in \( S \) as \( |S / G| \). Recall the Orbit-Stabilizer Theorem, which states that for any \( s \in S \):
\[
|G| = |\text{Orb}(s)| \cdot |\text{Stab}(s)|,
\]
where:
\[
\text{Orb}(s) = \{g \cdot s \mid g \in G\} \quad \text{and} \quad \text{Stab}(s) = \{g \in G \mid g \cdot s = s\}.
\]

Next, let's rewrite the sum \( \sum_{x \in G} f(x) \). This sum counts the total number of pairs \((g, s)\) such that \( g \cdot s = s \):
\[
\sum_{x \in G} f(x) = \# \{(g, s) \mid g \in G, s \in S, g \cdot s = s\} = \sum_{s \in S} |\text{Stab}(s)|.
\]

We now express this sum using the Orbit-Stabilizer Theorem:
\[
\sum_{s \in S} |\text{Stab}(s)| = \sum_{s \in S} \frac{|G|}{|\text{Orb}(s)|}.
\]

Since the set \( S \) is partitioned into disjoint orbits under the action of \( G \), we can rewrite the above sum as:
\[
\sum_{s \in S} \frac{|G|}{|\text{Orb}(s)|} = |G| \sum_{C \in S / G} \sum_{s \in C} \frac{1}{|\text{Orb}(s)|}.
\]

Each inner sum over \( s \in C \) simplifies to 1 because each element in an orbit contributes exactly \(\frac{1}{|\text{Orb}(s)|}\) for each element of \( G \) fixing \( s \). Hence, we have:
\[
\sum_{C \in S / G} \sum_{s \in C} \frac{1}{|\text{Orb}(s)|} = |S / G|.
\]

Therefore, we conclude:
\[
|S / G| = \frac{1}{|G|} \sum_{g \in G} f(g).
\]

This completes the proof that the number of orbits of \( G \) in \( S \) is given by \( \frac{1}{|G|} \sum_{g \in G} f(g) \).


\section*{Problem 20}
Let \( P \) be a \( p \)-group. Let \( A \) be a normal subgroup of order \( p \). Prove that \( A \) is contained in the center of \( P \).

\subsection*{Solution}
Given that \( P \) is a \( p \)-group and \( A \triangleleft P \) with \( |A| = p \), we want to show that \( A \subseteq Z(P) \), where:
\[
Z(P) = \{z \in P \mid \forall p \in P, \ zp = pz \}.
\]

Since \( P \) is a \( p \)-group, every element of \( P \) has order a power of \( p \). Given that \( A \triangleleft P \) and \( |A| = p \), we conclude that \( A \) is cyclic of order \( p \).

Let \( a \in A \) and \( p \in P \). By normality of \( A \), we have \( p a p^{-1} \in A \). Since \( A \) is cyclic of order \( p \), it is generated by \( a \). Therefore, \( p a p^{-1} \) must be of the form \( a^k \) for some integer \( k \). 

However, since the only automorphisms of a cyclic group of prime order \( p \) are the identity and the map sending each element to its inverse, the map \( x \mapsto p x p^{-1} \) must be the identity. Thus:
\[
p a p^{-1} = a.
\]

This implies \( p a = a p \). Therefore, for every \( a \in A \), \( a \) commutes with every element of \( P \). Hence, we have:
\[
A \subseteq Z(P).
\]
\section*{Problem 24}
Let \( p \) be a prime number. Show that a group of order \( p^2 \) is abelian, and that there are only two such groups up to isomorphism.

\subsection*{Solution}
We will first rewrite the statement in a form that is easier to check: We will show that a group of order \( p^2 \) is abelian and is isomorphic to either a cyclic group of order \( p^2 \) or a direct product of two cyclic groups of order \( p \).

Let \( G \) be a group of order \( p^2 \). We know that \( G \) is a \( p \)-group, so it must have a nontrivial center. Since \( Z(G) \) is a subgroup of \( G \), it must have order \( p^2 \) or \( p \). 
\\
- If \( |Z(G)| = p^2 \), then \( Z(G) = G \), implying that \( G \) is abelian.\\
- If \( |Z(G)| = p \), then \( |G / Z(G)| = \frac{|G|}{|Z(G)|} = \frac{p^2}{p} = p \) by Lagrange’s Theorem. 

Thus, \( G / Z(G) \) is cyclic and abelian.

Let \( gZ(G) \) be the generator of \( G / Z(G) \). Then any element \( x \in G \) can be expressed as \( x = g^n z \) for some \( z \in Z(G) \). Let \( x = g^m z_1 \) and \( y = g^n z_2 \) for some \( z_1, z_2 \in Z(G) \). Then:
\[
xy = (g^m z_1)(g^n z_2) = g^m g^n z_1 z_2 = g^{m+n} z_1 z_2 = g^n g^m z_2 z_1 = (g^n z_2)(g^m z_1) = yx.
\]
Therefore, \( G \) is always abelian.

By Sylow’s Theorem, a group \( G \) of order \( p^2 \) has at most one subgroup of order \( p \) and \( p^2 \). If \( G \) is generated by a single element, then it is cyclic and is isomorphic to \( \mathbb{Z} / p^2\mathbb{Z} \).

If \( G \) is not cyclic, it must contain elements of order \( p \) except the identity. By Sylow's Theorem, there exists a subgroup \( H \) of order \( p \). Let \( x \in G \) be an element of order \( p \), and \( H = \langle x \rangle \). Let \( y \in G \) be another element of order \( p \), and let \( K = \langle y \rangle \). Since \( H \cap K = \{e\} \), we have:
\[
|HK| = \frac{|H||K|}{|H \cap K|} = p^2 \implies G \cong H \times K \cong \mathbb{Z} / p\mathbb{Z} \times \mathbb{Z} / p\mathbb{Z}.
\]

Thus, the only two groups of order \( p^2 \) up to isomorphism are \( \mathbb{Z} / p^2\mathbb{Z} \) and \( \mathbb{Z} / p\mathbb{Z} \times \mathbb{Z} / p\mathbb{Z} \).


\section*{Problem 26}

(a) Let \( G \) be a group of order \( pq \), where \( p, q \) are primes with \( p < q \). Assume that \( q \not\equiv 1 \mod p \). Prove that \( G \) is cyclic.

(b) Show that every group of order 15 is cyclic.

\subsection*{Solution}
\subsubsection*{(a)}
Let \( G \) be a group with \( |G| = pq \), where \( p, q \) are primes and \( p < q \) with \( q \not\equiv 1 \mod p \).

Let \( n_p \) be the number of Sylow \( p \)-subgroups of \( G \). By Sylow’s theorems, we have:
\[
n_p \equiv 1 \mod p \quad \text{and} \quad n_p \mid q.
\]
Similarly, let \( n_q \) be the number of Sylow \( q \)-subgroups of \( G \). Then:
\[
n_q \equiv 1 \mod q \quad \text{and} \quad n_q \mid p.
\]

Since \( p < q \), it follows that \( n_p = 1 \) and \( n_q \) can be either \( 1 \) or \( p \). Given \( n_q \equiv 1 \mod q \), we conclude that \( n_q = 1 \).

Since there exist unique subgroups \( H \) of order \( p \) and \( K \) of order \( q \), we have \( H \triangleleft G \) and \( K \triangleleft G \). 

Both \( H \) and \( K \) are groups of prime order, making them cyclic and abelian, with \( H \cap K = \{e\} \). 

By the previous problem's result, we know \( G \cong H \times K \cong \mathbb{Z}/p\mathbb{Z} \times \mathbb{Z}/q\mathbb{Z} \). Consider the element \( (1, 1) \in H \times K \), where 1 denotes the generator of each cyclic group. The order of this element is given by:
\[
\text{lcm}(\text{order of } p, \text{ order of } q) = \text{lcm}(p, q) = pq.
\]
Since \( G \) has an element of order \( pq \), it follows that \( G \) is cyclic.

\subsubsection*{(b)}
Let \( G \) be a group with \( |G| = 15 \). Since \( |G| = 3 \cdot 5 \), with 3 and 5 being primes, we have:
\[
5 \not\equiv 1 \mod 3.
\]
By part (a), \( G \) must be cyclic. Hence, every group of order 15 is cyclic.

\end{document}
